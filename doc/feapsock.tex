\section{Reaping child processes}

In UNIX, child processes are not released to the system until after the
parent process checks the child process exit status using {\tt wait} or
{\tt waitpid}.  We install a handler on SIGCHLD events (change of child
status) that checks the exit status of any child processes that are 
finished.

This is a standard piece of most UNIX daemons.

\begin{verbatim}
static void sigchld_handler(int s)
{
    while (waitpid(-1, NULL, WNOHANG) > 0);
}

static void install_reaper()
{
    struct sigaction sa;
    sa.sa_handler = sigchld_handler;
    sigemptyset(&sa.sa_mask);
    sa.sa_flags = SA_RESTART;
    ec(sigaction(SIGCHLD, &sa, NULL));
}

\end{verbatim}
\section{Receiving TCP socket connections}

On the server side, there are two phases to setting up a socket connection.
First, we need to set up a socket -- create it, give it an address with
{\tt bind}, and use {\tt listen} to tell the system that we can receive
connections on it.  By default, the server listens on {\tt MYPORT} (3490),
but this value can be changed by setting the {\tt MATFEAP\_PORT} environment
variable.

\begin{verbatim}
static int tcp_socket_setup(int port)
{
    int sockfd;                           /* socket file descriptor */
    struct sockaddr_in my_addr;           /* my address information */
    int yes=1;

    memset(&my_addr, 0, sizeof(my_addr));
    my_addr.sin_family = AF_INET;
    my_addr.sin_port = htons(port);
    my_addr.sin_addr.s_addr = INADDR_ANY;

    ec(sockfd = socket(AF_INET, SOCK_STREAM, 0));
    ec(setsockopt(sockfd, SOL_SOCKET, SO_REUSEADDR, &yes, sizeof(int)));
    ec(bind(sockfd, (struct sockaddr*) &my_addr, sizeof(my_addr)));
    ec(listen(sockfd, BACKLOG));

    printf("Server listening on port %d\n", port);
    return sockfd;
}

static int tcp_handle_connection(int sockfd)
{
    struct sockaddr_in their_addr;
    socklen_t sin_size = sizeof(struct sockaddr_in);
    while (1) {
        int new_fd = accept(sockfd, (struct sockaddr*) &their_addr, &sin_size);
        if (new_fd < 0)
            perror("accept");
        else {
            time_t c = time(NULL);
            printf("Connection from %s -- %s",
                   inet_ntoa(their_addr.sin_addr), ctime(&c));
            return new_fd;
        }
    }
}

\end{verbatim}
\section{Receiving local socket connections}

In addition to TCP socket connections, we allow the server to use
UNIX domain socket connections.  UNIX domain sockets are used in various
other system servers as well, including X11.  The primary advantage of
UNIX-domain sockets over TCP sockets is performance: if you're
going to run both the client and the server on the same machine and
you use a UNIX-domain socket, then the operating system can handle
context switches a little more intelligently.  The disadvantage of
using UNIX domain sockets is that Java doesn't know about them at
this time -- you have to use the MEX-based socket infrastructure to
connect to the UNIX domain server.

UNIX domain sockets differ from TCP sockets primarily in the setup
phase -- afterward, everything works the same.  The location of a
UNIX domain socket is specified as a filesystem location rather than
a port number.  We use the existence of a port name to tell MATFEAP
to listen on a UNIX domain socket.

\begin{verbatim}
static int local_socket_setup(const char* sockname)
{
    int sockfd;                           /* socket file descriptor */
    struct sockaddr_un my_addr;           /* my address information */
    int len;

    /* Remove any previous socket */
    unlink(sockname);

    memset(&my_addr, 0, sizeof(my_addr));
    my_addr.sun_family = AF_UNIX;
    strcpy(my_addr.sun_path, sockname);
    len = sizeof(my_addr.sun_family) + strlen(my_addr.sun_path) + 1;

    ec(sockfd = socket(AF_UNIX, SOCK_STREAM, 0));
    ec(bind(sockfd, (struct sockaddr*) &my_addr, len));
    ec(listen(sockfd, BACKLOG));

    printf("Server listening on local socket %s\n", sockname);
    return sockfd;
}

static int local_handle_connection(int sockfd)
{
    struct sockaddr_un their_addr;
    socklen_t sin_size = sizeof(struct sockaddr_un);
    while (1) {
        int new_fd = accept(sockfd, (struct sockaddr*) &their_addr, &sin_size);
        if (new_fd < 0)
            perror("accept");
        else {
            time_t c = time(NULL);
            printf("Connection -- %s", ctime(&c));
            return new_fd;
        }
    }
}

\end{verbatim}
\section{Deciding on a socket connections}

We decide whether to use TCP or UNIX domain sockets based on
the setting of the environment variables.  If
{\tt MATFEAP\_SOCKNAME} is set, we use that as the address for
a local UNIX-domain socket.  Otherwise, if {\tt MATFEAP\_PORT} is
set, we use that as the port number for a TCP-based socket.
If no relevant environment variable is set, then we default to
a TCP-based server listening on port 3490.

\begin{verbatim}
static int feapsock_local_socket;

static int socket_setup()
{
    int port = MYPORT;
    char* port_env = getenv(PORT_ENV_VAR);
    char* sockname = getenv(SOCKNAME_ENV_VAR);
    if (port_env)
        port = atoi(port_env);

    feapsock_local_socket = (sockname != 0);
    if (feapsock_local_socket)
        return local_socket_setup(sockname);
    else
        return tcp_socket_setup(port);
}

static int handle_connection(int sockfd)
{
    if (feapsock_local_socket)
        return local_handle_connection(sockfd);
    else
        return tcp_handle_connection(sockfd);
}

\end{verbatim}
\section{Redirecting I/O streams}

UNIX treats socket file descriptors like any other file
descriptors.  That means the socket input and output streams can be
connected to {\tt stdin} (fd 0) and {\tt stdout} (fd 1) using the
{\tt dup} or {\tt dup2} system calls.  I leave {\tt stderr} alone
so that the FEAP server can send debugging information to the terminal
without breaking the protocol used to communicate between the server and
the client.

\begin{verbatim}
static void send_std_to_socket(int new_fd)
{
    dup2(new_fd, 0);
    dup2(new_fd, 1);
    /*dup2(new_fd, 2);*/
    close(new_fd);
}

\end{verbatim}
\section{The main daemon}

The main loop is a Fortran-callable routine that accepts incoming
socket connections from clients and assigns to each a simulation
process.  The call to {\tt feapserver} in the original process
never exits.  In child processes created to handle incoming
connections, {\tt feapserver} returns control to the calling routine,
allowing FEAP to continue running as it usually would.

\begin{verbatim}
int feapserver_()
{
    int sockfd = socket_setup();
    install_reaper();

    while (1) {
        int new_fd = handle_connection(sockfd);
        if (!fork()) {  /* This is the child process */
            close(sockfd);
            send_std_to_socket(new_fd);
            return 0;
        }
        close(new_fd);  /* Parent doesn't need this */
    }

    exit(0);
}

\end{verbatim}
